\documentclass[a4paper,9pt,journal]{IEEEtran}
\usepackage{graphicx} \usepackage[cmex10]{amsmath}
\usepackage{amssymb} \usepackage{fancybox} \usepackage{alltt}
\usepackage{soul} \usepackage{color} \usepackage{verbatim}
\usepackage{xcolor} \usepackage{colortbl,hhline}
\usepackage[ruled,vlined]{algorithm2e} \usepackage{framed}
\usepackage{amsthm} \usepackage{fancyref}
\usepackage{amsmath,amsfonts,amssymb, amsthm} \usepackage{textcomp}

\newtheorem{remark}{Remark}

\def \lb {{\langle}} \def \rb {{\rangle}}
\newcommand{\fro}[1]{\|#1\|_2}
\newcommand{\theHalgorithm}{\arabic{algorithm}}

\newcommand{\argmin}{\mathop{\mathrm{argmin}}}

\usepackage{hyperref}


\newtheorem{theorem}{Theorem} \newtheorem{lemma}[theorem]{Lemma}
\newtheorem{proposition}[theorem]{Proposition}
\newtheorem{corollary}[theorem]{Corollary}
\newtheorem{definition}[theorem]{Definition}


\newcommand{\prox}{\textrm{pr ox} } \newcommand{\grad}{\textrm{grad} }
\newcommand{\dive}{\textrm{div} }

\pagenumbering{gobble}

\begin{document}
\onecolumn
\title{A fast and simple primal-dual algorithm for computing Nash equilibria and best response strategies
  in two-person zero-sum sequential games with imcomplete information}


\author{\IEEEauthorblockN{DOHMATOB Elvis}}


\maketitle

\begin{abstract}
% LESS THAN 200 WORDS !!!!!
% 
% While medical imaging typically provides massive amounts of
% data, the extraction of relevant information in a given
% applicative context remains a difficult challenge.
% XXX : don't call it SPM but just "predictive regions or predictive
% maps"
In this manuscript we present a simple primal-dual algorithm for computing Nash equilibria and best reponse strategies in  sequential games with
imcomplete information and perfect recall. The proposed algorithm derives from the primal-dual scheme of A. Chambolle and T. Pock, applied to the
equilibrium and best response strategy problems for the sequence form representation of the game. We use a dualization trick to render all
proximity operators computable in closed-form.

%% In this manuscript, we consider the problem of computing a best response against an opponent's realization plan in two-person sequential games.
%% The proposed algorithm for solving the corresponding contrained convex-optimization problem, derives from the primal-dual scheme of A. Chambolle and T. Pock.
%% Our algorithm is simple: all resolvent operators can be effectively computed in closed-form, using only elementary algebraic operations.
%%  and is orders of magnitude more efficient than state-of-the-art algorithms like those based on Linear Programming, Interior-Point methods,
%% and more recently, the Nesterov Excessive Gap Technique.
% By way of example, we conclude by exhibiting results on Texas Hold'em Poker.
\end{abstract}


\begin{IEEEkeywords}
  game theory; Nash equilibrium; two-person sequential game; imcomplete information; perfect recall; sequence-form representaion; best response strategy;
  convex-optimization; dualization; primal-dual algorithms
\end{IEEEkeywords}

\section{Introduction}
%\label{sec:intro}
\subsection{Notation and Terminology}
Give positive integers $m$ and $n$, $\mathbb{R}^{n}$ denotes the $n$-dimensional real vector space, and $\mathbb{R}^{m \times n}$
the space of all $m$-by-$n$ real matrices. $0_{m,n}$ denotes the $m$-by-$n$ matrix of zeros.
$\mathbb{R}^n_+$ := $\{x \in \mathbb{R}^n|x_j \ge 0 \text{ }  \forall j\}$ is the $n$-dimensional \textit{nonnegati
ve orthant}.
For a vector $x \in \mathbb{R}^n$, $\|x\|$ denotes the $2$-\textit{norm} of $x$ defined by $\|x\| := \sqrt{x^Tx}$.
$(x)_+$ denotes its point-wise maximum with 0. Note that $(x)_+ \in \mathbb{R}^n_+$.
For example, $((-2, \pi))_+ = (max(-2, 0), max(\pi, 0)) = (0, \pi)$. The \textit{spectral norm} of a matrix $K$, denoted $\|K\|$, is defined by
$\|K\| := \underset{x \ne 0}{max}\text{ }\frac{\|Kx\|}{\|x\|}$. Equivalently, $\|K\|$ is the largest \textit{singular value} of $K$, i.e the largest
\textit{eigen-value} of $K^TK$. Given a convex subset $C$ of $\mathbb{R}^n$, $i_C$ denotes its \textit{indicator function} defined by
$i_C(x) = 0$ if $x \in C$ and $i_C(x) = +\infty$ otherwise. Finally, given a \textit{proper} convex function (i.e a convex function which takes a finite
value at least a one point) $f : \mathbb{R}^n \rightarrow [0, +\infty]$, and a positive real number $\tau$, the \textit{proximal operator} of $f$ of rank $\tau$,
denoted $\text{prox}_{\tau f}$ is the function which maps a point $x \in \mathbb{R}^n$ to the (necessarily unique) solution of the problem
\begin{equation}
  \underset{z \in \mathbb{R}^n}{argmin}\text{ }\frac{1}{2}\|z - x\|^2 + \tau f(z)
\end{equation}

%% We will be needing the following notation: %%  The reader should lookup any standard textbook
%% %% (for example \cite{boyd2004}) on convex optimization for a tutorial introduction to these notions.
%% Viz,
%% \begin{itemize}
%% \item $\mathbb{R}^n$: $n$-dimensional real vector space;
%% \item $\mathbb{R}^{m \times n}$: \quad space of all $m$-by-$n$ real matrices;
%% \item $0_{m,n}$: $m$-by-$n$ matrix of zeros;
%% \item $(x)_+$: \quad component-wise maximum of a vector $x$ and 0;
%% \item $\mathbb{R}^n_+$: \quad $\{x \in \mathbb{R}^n|x = (x)_+\}$, the $n$-dimensional nonnegative orthant 
%% \item $i_C$: \quad indicator function of a convex set $C$;
%% % \item $\Pi_C$: \quad euclidean projector onto a convex set $C$;
%% \item $\|K\|_2$: \quad spectral norm of a matrix $K$
%% % \item $F^*$: the convex conjugate of a convex function $F$.
%% %% \item \textit{l.s.c.p.c}: \quad acronym for adjective \textit{lower semi-continuous proper convex};
%% %% \item $f^*$: \quad Fenchel transform (a.k.a convex conjugate) of a \textit{l.s.c.p.c} function $f$;
%% \end{itemize}

\subsection{Statement of the problem}
We are interested in the problem of computing a Nash equilibrium for a two-person zero-sum sequential game with imcomplete information
and perfect recall (e.g Texas Hold'em Poker, etc.). In the \textit{sequence-form representation} of the game, the problem has the following saddle-point formulation
\begin{equation}
  \underset{y \in Q_2}{minimize}\text{ }\underset{x \in Q_1}{maximize}\text{ }{y^TAx}
  \label{eq:opt_pb}
\end{equation}

where $Q_j := \{z \in \mathbb{R}_+^{n_j}|E_jz=e_j\}$, for $j \in \{1, 2\}$. As usual, the ``minimize-maximize'' notation in problem
\eqref{eq:opt_pb} means that a pair $(x^*, y^*) \in Q_1 \times Q_2$ is a solution iff
\begin{equation}
  {y^*}^TAx \le y^TAx \le y^TAx^*, \forall (x, y) \in Q_1 \times Q_2
\end{equation}
Such pairs correspond to the Nash equilibria of the game.

We recall that $E_j$ is a $p_j$-by-$n_j$ matrix whose
entries are $-1$, $0$, or $+1$, and $e_j := (1, 0, 0, ..., 0)$ is a vector of length $p_j$. We also recall that $E_1$ and $e_1$ (resp. $E_2$ and $e_2$)
encode linear constraints on our (resp. the opponent's)  ``admissible'' realization $x$ (resp. $y$). $A$ is the payoff matrix, of size $n_2 \times n_1$.


\begin{remark}
  In practice, the matrices $A$, $E_1$, and $E_2$ are large but very sparse too.
%% : $A$ will be sparse because a concrete sequential game will
%% typically have very few\footnote{Few, relative to the size of the game tree.} leafs, and only a few
%% combinations of sequences of moves of the players, will actually lead to a leaf (i.e. end the game);
%% $E$ and $F$ will be sparse because the kinks of possible sequences of moves of each player will
%% zig-zag between only a limited number of the player's information sets so that a move at an information set will
%% rarely\footnote{Relative to the number of information sets for the player.}  extend another information set.
This sparsity should be thoroughly exploited by a solver for problem \eqref{eq:opt_pb}.
\end{remark}

%% In section \ref{sec:related_work}, we give a brief overview of existing methods for solving \eqref{eq:opt_pb}.
%% We elaborate our proposed algorithm in section \ref{sec:algo}.

%% \section{Related work}
%% \label{sec:related_work}

%% Pending...

\section{The proposed algorithm}
We preprose to solve problem \eqref{eq:opt_pb} using the primal-dual scheme of A. Chambolle and T. Pock.
Though this scheme has recently gained considerable popularity in the signal processing community, to the best of our knowledge,
this is the first time it is being applied to game theoretic problems.

%% \label{sec:algo}
%% In this section we present the algorithm which is the purpose of this paper, namely an algorithm
%% for solving \eqref{eq:opt_pb}. Our algorithm (Alg.\ref{Tab:algo}) is a use-case of
%% the generic primal-dual algorithm of A. Chambolle and T. Pock, namely Algorithm 1 of \cite{chambolle2010}.

Now, observe that
\begin{eqnarray*}
  i_{Q_j}(z) = i_{\mathbb{R}^{n_j}_+}(z) + \underset{w \in \mathbb{R}^{n_j}}{max}\text{}{w^T(E_jz - e_j)}, \forall z \in \mathbb{R}^{n_j},
\end{eqnarray*}
and so problem \eqref{eq:opt_pb} can be re-written as

\begin{equation}
  \underset{y \in \mathbb{R}^{n_2}, v\in \mathbb{R}^{p_1}}{minimize}\text{ }\underset{x \in \mathbb{R}^{n_1}, u \in \mathbb{R}^{p_2}}{maximize}
           {\begin{bmatrix}y\\v\end{bmatrix}^TK\begin{bmatrix}x\\u\end{bmatrix} + G(y, v) - F(x, u)}
  \label{eq:my_opt_pb}
\end{equation}

where:
\begin{equation}
  \left .
  \begin{split}
    K :=
    \left[
      \begin{array}{c|c}
        A & E_1^T \\ \hline
        E_2 & 0_{p_2, p_1}
      \end{array}
      \right] \in \mathbb{R}^{(n_2 + p_1) \times (n_1 + p_2)} \\
    %%\begin{bmatrix}A \text{ } E_1^T\\ E_2 \text{ } 0\end{bmatrix} \in \mathbb{R}^{(n_2 + p_1) \times (n_1 + p_2)}\\
      G: \mathbb{R}^{n_2} \times \mathbb{R}^{p_1} \rightarrow [0, +\infty], (y, v) \mapsto i_{\mathbb{R}^{n_2}_+}(y) - v^Te_1\\
      F: \mathbb{R}^{n_1} \times \mathbb{R}^{p_2} \rightarrow [0, +\infty], (x, u) \mapsto i_{\mathbb{R}^{n_1}_+}(x) - u^Te_2
  \end{split}
  \right\}
\end{equation}

It is clear that $G$ and $F$ are proper convex lower semi-continuous and a straightforward computation reveals that their proximal operators are given by
\begin{equation}
  \left .
  \begin{split}
    (1 + \tau \partial G)^{-1}: \mathbb{R}^{n_2} \times \mathbb{R}^{p_1} \rightarrow \mathbb{R}^{n_2} \times \mathbb{R}^{p_1}, (y, v) \mapsto ((y)_+, v + \tau e_1)\\
    (1 + \sigma \partial F)^{-1}: \mathbb{R}^{n_1} \times \mathbb{R}^{p_2} \rightarrow \mathbb{R}^{n_1} \times \mathbb{R}^{p_2}, (x, u) \mapsto ((x)_+, u + \sigma e_2)
  \end{split}
  \right\}
\end{equation}

%% \begin{algorithm}[htb]
%%   \caption{Primal-dual algorithm for computing best response against opponent's fixed realization plan $y_0$}
%%   \textbf{Given} Tolerance $\epsilon > 0$.\\
%%   \textbf{Initialize} $\tilde{x^{(0)}} = x^{(0)} \in \mathbb{R}^n$; $\zeta^{(0)} \in \mathbb{R}^{q}$;
%%   $\tau, \sigma > 0 \text{ s.t. }\tau\sigma \|K\|_2^2 < 1$; $k = 0$.\\
%%   \textbf{Precompute} $\eta_0 \leftarrow \tau A^Ty_0$\\
%%   \Repeat{
%%     $\frac{\|x^{(k+1)} - x^{(k)}\|^2_2}{2\tau} + \frac{\|\zeta^{(k+1)} - \zeta^{(k)}\|^2_2}{2\sigma} < \epsilon$}{
%%     \begin{eqnarray*}
%%       \zeta^{(k+1)} &\leftarrow& \zeta^{(k)} - \sigma \bigl(e - E\tilde{x}^{(k)}\bigr)\\
%%       x^{(k+1)} &\leftarrow& \bigl(x^{(k)} + \eta_0 + E^T\zeta^{(k+1)}\bigr)_+\\
%%       \tilde{x}^{(k+1)} &\leftarrow& 2x^{(k+1)} - x^{(k)}\\
%%       k &\leftarrow& k + 1
%%     \end{eqnarray*}
%%   } \Return $x^{(k)}$, $-y_0^TAx^{(k)}$
%%   \label{Tab:algo}
%% \end{algorithm}

%% \subsection{Convergence analysis of the algorithm}
%% Pending...

%% \begin{remark}
%% The derivation above reveals that in equation \eqref{eq:opt_pb} above, if the constraint ``$x \ge 0$'' is replaced by
%% a constraint ``$x \in C$'' (thus obtaining a new problem) where $C$ is a convex set
%% %% onto which euclidean projections
%% %% can be easily computed (for example, a probability simplex)
%% , then we simply need to replace the operator ``$(.)_+$''
%% with ``$\Pi_C$'' in the equations to obtain a corresponding algorithm. Of course, this is because $i_C^* = \Pi_C$.
%% \end{remark}

%% \medskip \noindent
%% \textbf{Acknowledgments:}
%% Pending...

% ==========
% = biblio =
% ==========
% {\small
\bibliographystyle{IEEEtran} \bibliography{IEEEabrv,bib_tv.bib,agt.bib}
\end{document}




%
%


