% \documentclass[a4paper,11pt]{article}
% \documentclass[a4paper,10pt,draftcls,journal]{IEEEtran}
\documentclass[a4paper,10pt,journal]{IEEEtran}
% \usepackage{times}
\usepackage{graphicx} \usepackage[cmex10]{amsmath}
\usepackage{amssymb} \usepackage{fancybox} \usepackage{alltt}
\usepackage{soul} \usepackage{color} \usepackage{verbatim}
\usepackage{xcolor} \usepackage{colortbl,hhline}
%\usepackage{algorithm2e}
% \usepackage{algorithm}
\usepackage[ruled,vlined]{algorithm2e} \usepackage{framed}
\usepackage{amsthm} \usepackage{fancyref}
\usepackage{amsmath,amsfonts,amssymb, amsthm} \usepackage{textcomp}
\newtheorem*{remark}{Remark}
 
% \usepackage{natbib}
% \bibpunct{[}{]}{;}{a}{,}{,}

\def \lb {{\langle}} \def \rb {{\rangle}}
\newcommand{\fro}[1]{\|#1\|_2}
\newcommand{\theHalgorithm}{\arabic{algorithm}}

\newcommand{\argmin}{\mathop{\mathrm{argmin}}}

\usepackage{hyperref}


\newtheorem{theorem}{Theorem} \newtheorem{lemma}[theorem]{Lemma}
\newtheorem{proposition}[theorem]{Proposition}
\newtheorem{corollary}[theorem]{Corollary}
\newtheorem{definition}[theorem]{Definition}


\newcommand{\prox}{\textrm{pr ox} } \newcommand{\grad}{\textrm{grad} }
\newcommand{\dive}{\textrm{div} }

\pagenumbering{gobble}

\begin{document}
%\onecolumn
\title{Primal-dual algorithm for computing best response against the unplanned mixed strategies of an opponent in two-person zero-sum sequential games}


\author{\IEEEauthorblockN{Elvis D. DOHMATOB}}


\maketitle

\begin{abstract}
% LESS THAN 200 WORDS !!!!!
% 
% While medical imaging typically provides massive amounts of
% data, the extraction of relevant information in a given
% applicative context remains a difficult challenge.

% XXX : don't call it SPM but just "predictive regions or predictive
% maps"
In this manuscript, we consider the problem of computing the best response against the  unplanned mixed strategies of an opponent
in two-player zero-sum sequential games. An unplanned mixed strategy is simply a strategy chosen from a probability simplex.
The proposed technique employs the primal-dual scheme of A. Chambolle and T. Pock to solve the corresponding saddle-point problem.
The resulting algorithm is simple (all resolvent operators can be effectively computed quasi closed-form, using only elementary algebraic operations)
 and is orders of magnitude more efficient than state-of-the-art algorithms like those based on Linear Programming, Interior-Point methods,
and more recently, the Nesterov Excessive Gap Technique. The problem considered here encompasses the problem of computing the best response against an opponent's fixed strategy.
We conclude by exhibiting results on Texas Hold'em Poker.
\end{abstract}


\begin{IEEEkeywords}
  game theory; two-person zero-sum sequential game; best response; mixed strategy; simplex; convex-optimization; saddle-point problem; primal-dual scheme
\end{IEEEkeywords}

\section{Introduction: Statement of the problem}
\label{sec:intro}
%% Let $A \in \mathbb{R}^{m,n}$ be the payoff matrix; let $E \in \mathbb{R}^{r,n}$ and $e \in \mathbb{R}^r$ be a matrix so that player 1 wishes only to use those strategies  $x \in \mathbb{R}_{+}^n$ for which $Ex = e$, i.e $x \in \mathbb{R}_{+}^n$
%%  %% from player \textbf{row} perspective, so that player \textbf{row} gains $A_{i,j}$ and player \textbf{column} forfeits $A_{i,j}$ euros if the former plays the $i$th row and the former players the $j$th column.

%% We denote by $\mathbf{y} \in \mathbb{R}^{n}$ the targets to
%% be predicted, and $\mathbf{X}\in\mathbb{R}^{n \times p}$ the
%% brain images related to the presentation of different
%% stimuli. $p$ is the number of voxels and $n$ the number of
%% samples (images). Typically, $p \sim 10^3-10^5$ (for a whole volume),
%% while $n \sim 10-10^3$.  Let $\Omega \subset \mathbb{R}^3$ be
%% the 3D image domain, discretized on a finite grid. The coefficients
%% $\mathbf{w}$ define a spatial map in $\mathbb{R}^p$. Its gradient at a
%% voxel $j \in \Omega $ reads\footnote{Where it is understood that the feature vector $w$ is ``unmasked''
%% into a full 3D volume before the spatial operators are applied.}:
%% \begin{eqnarray}
%%   \mathbf{\nabla} w(j) := [\nabla_{x} w(j),
%%     \nabla_{y} w(j), \nabla_{z} w(j)] \in
%%   \mathbb{R}^3,
%% \end{eqnarray}
%% where $\nabla_{u}$ is the spatial difference operator along
%% the $u$-axis. Thus $\mathbf{\nabla}$
%% is a linear operator $\in \mathbb{R}^{3p \times
%%   p}$ with adjoint $\mathbf{\nabla} ^T = -\dive \in \mathbb{R}^{p
%%   \times 3p}$. Let $\tilde{\nabla} := [(1 - \rho)\nabla^T, \rho I]^T$ be the identity-augmented spatial gradient operator. Finally, let us recall the definition of the $(q, 1)$ mixed-norm\footnote{Where $q$ itself is allowed to be a mixed-norm.}:
%% \begin{equation}
%% \|v\|_{q, 1} := \sum_{j}{\|v_{j}\|_{q}}
%% \end{equation}

Consider a two-person zero-sum sequential game with payoff matrix $A \in \mathbb{R}^{m,n}$. Consider the problem of
finding the best response strategy $x^* \in \mathbb{R}_{+}^n \cap E^{-1}(\{e\})$ for player 1, given a fixed
(possibly mixed) strategy $y_0$ for player 2, where $E \in \{-1, 0, +1\}^{r,n}$ and $(1, 0, 0, ..., 0) =: e \in \mathbb{R}^r$ encode player
1's game plan, consistent with the rules of the game. In the language of convex-optimization, one can readily give the following saddle-point formulation for this problem. Viz,

\begin{equation}
  \underset{x \in \mathbb{R}_{+}^n \cap E^{-1}(\{e\})}{min}\text{ }\underset{y \in \Delta_{m}^0}{max}\text{ }y^TAx
\end{equation}

where $\Delta_{m}^0$ is the singleton $\{y_0\}$ if $y_0$ is known, otherwise it is the $m-$dimensional probability simplex $\Delta_{m} := \{x \in \mathbb{R}_{+}^m \text{ s.t }\textbf{1}^Tx = 1\}$.\\\\
Noting that $i_{E^{-1}(\{e\})}(x) = \underset{\eta \in \mathbb{R}^r}{max}\text{ }\eta^T(e - Ex)$, one can re-write (1) as:
\begin{equation}
  \underset{x \in \mathbb{R}^n}{min}\text{ }\underset{y \in \mathbb{R}^m, \eta \in \mathbb{R}^r}{max}\text{ }[y,\eta]^T\left[ \begin{array}{@{}cc@{}}
    A \\ \hline
    -E
    \end{array} \right]x + G(x) - F^*(y, \eta)
\end{equation}
where:
\begin{equation}
  \begin{aligned}
    G:=i_{\mathbb{R}_{+}^n}\\
    F^*: (y, \eta) \mapsto (i_{\Delta_{m}^0}(y) + e^T\eta)
  \end{aligned}
\end{equation}
with resolvent operators given by
\begin{equation}
  \begin{aligned}
    (1 + \tau \partial G)^{-1}: x \mapsto \Pi_{\mathbb{R}_{+}^n}(x) = (x)_+\\
    (1 + \sigma \partial F^*)^{-1}: (y, \eta) \mapsto (\Pi_{\Delta_{m}^0}(y), \eta - \sigma e)% (z, \eta) \mapsto \bigl(y, \frac{\eta}{1 + \sigma}\bigr)
  \end{aligned}
\end{equation}
This way, we obtain following algorithm which solves (2).

\begin{algorithm}[htb]
\caption{Primal-dual algorithm for computing best response against on plained mixed strategies of opponent}%%  \textbf{Let}
\textbf{Initialize} $\tilde{x}^{(0)} = x^{(0)} \in \mathbb{R}^n$; $y^{(0)} \in \mathbb{R}^{m}$; $\eta^{(0)} \in \mathbb{R}^{m}$;
$\tau, \sigma > 0 \text{ s.t. }\tau\sigma (\|A\|^2 + \|E\|^2) \le 1$; $k = 0$.\\
\Repeat{
  convergence}{
  \begin{eqnarray*}
    y^{(k+1)} &\leftarrow& \Pi_{\Delta_{m}^0}\bigl(y^{(k)} + \sigma A \tilde{x}^{(k)}\bigr)\\
    \eta^{(k+1)} &\leftarrow& \eta^{(k)} - \sigma E \tilde{x}^{(k)} - \sigma e\\
    x^{(k+1)} &\leftarrow& (x^{(k)} - \tau A^Ty^{(k+1)} + E^T\eta^{(k+1)})_+\\
    \tilde{x}^{(k+1)} &\leftarrow& 2x^{(k+1)} - x^{(k)}\\
    k &\leftarrow& k + 1
  \end{eqnarray*}
} \Return $x^{(k)}$
\label{Tab:pseudocode_lbfgs}
\end{algorithm}

%\begin{remark}
%% Due to the independence of $(1 + \sigma \partial F^*)^{-1}$ on $\sigma$, we can always choose the pair
%% $(\tau, \sigma)$ , such that $\tau\sigma \|\tilde{\nabla}\|^2 \le 1$ and $\tau \rho(X^TX) \le \alpha$.
% \end{remark}
%% We note that if the pair $(\tau, \sigma)$ is chosen such that $\tau \rho(X^TX) < \alpha$,
%% then (6) can be approximated by
%% \begin{equation}
%% (1 + \tau \partial G)^{-1}(w) \approx (1 - \frac{\tau}{\alpha}X^TX)(w - \frac{\tau}{\alpha} X^Ty)
%% \end{equation}

%% Prediction of external variates from brain images has seen an
%% explosion of interest in the past decade, in cognitive neurosciences
%% to predict cognitive content from functional imaging data such as fMRI
%% \cite{haxby2001,kamitani2005,haynes2006} or for medical diagnosis
%% purposes \cite{fan2008}. Given that brain images are
%% high-dimensional objects --composed of many voxels-- and the
%% number of samples is limited by the cost of acquisition, the
%% estimation of a multivariate predictive model is ill-posed and calls
%% for regularization. This regularization is central as it encodes the
%% practicioner's priors on the spatial maps. For brain mapping, it has
%% been shown that regularization schemes based on sparsity (Lasso or
%% $\ell_1$ family of models) \cite{carroll2009} or Total Variation (TV),
%% that promotes spatial contiguity \cite{michel2011}, perform well for
%% prediction. The combination of these, hereafter dubbed
%% ``TV$-\ell_{1}$'', extracts spatially-informed %% --and thus, more
%% %% interpretable-- 
%% brain maps that are more stable \cite{baldassarre2012}
%% and recover better the predictive regions
%% \cite{gramfort-etal:2013a}. In addition, this prior leads to
%% state-of-the-art methods for extraction of brain atlases
%% \cite{abraham2013}.

%% However, the corresponding optimization problem is
%% intrinsically hard to solve. The reason for this is two-fold. First and
%% foremost, in fMRI studies, the design matrix $\mathbf{X}$ is ``fat''
%% ($n \ll p$), dense, ill-conditioned with little algebraic structure to be
%% exploited, making the optimization problem ill-conditioned. Second, the
%% penalty is not smooth, and while the $\ell_{1}$ term is \emph{proximable}
%% (via \emph{soft-thresholding}), the TV term does not admit a closed-form proximal
%% operator. Thus neither gradient-based methods (like Newton, BFGS, etc.) nor proximal methods (like
%% ISTA~\cite{daubechies2004}, FISTA~\cite{beck2009a}) can be used in the
%% traditional way.

%% While the quality of the optimization may sound as a minor
%% technical problem to the practitioner, the sharpening effect of TV and the
%% sparsity-inducing effect of $\ell_1$ come into play only for
%% well-optimized solutions. As a result, the brain maps extracted vary
%% significantly as a function of the tolerance on the solver (see
%% Fig.\,\ref{fig:maps_tolerance}).

%% \begin{figure*}
%% \includegraphics[width=.32\linewidth]{maps/face_vs_house_tol_0_1.pdf}%
%% \llap{\color{white}\raisebox{.17\linewidth}{\rlap{\sffamily Stopping:
%%       $\Delta E < 10^{-1}$}}\hspace*{.315\linewidth}}\hfill%
%% \includegraphics[width=.32\linewidth]{maps/face_vs_house_tol_0_001.pdf}%
%% \llap{\color{white}\raisebox{.17\linewidth}{\rlap{\sffamily Stopping:
%%       $\Delta E < 10^{-3}$}}\hspace*{.315\linewidth}}\hfill%
%% \includegraphics[width=.32\linewidth]{maps/face_vs_house_tol_1e-05.pdf}%
%% \llap{\color{white}\raisebox{.17\linewidth}{\rlap{\sffamily Stopping:
%%       $\Delta E < 10^{-5}$}}\hspace*{.315\linewidth}}%

%% \caption{TV$-\ell_1$ maps for the face-house discrimination task on
%%   the visual recognition dataset, with regularization parameters chosen by
%%   cross-validation, for different stopping criteria.
%%   Note that
%%   the stopping criterion is defined as a threshold on the energy
%%   decrease per one iteration of the algorithm, and thus differs from
%%   the tolerance displayed in figure \ref{Fig:MSEtimes}.  This figure shows
%%   the importance of convergence for problem \eqref{eq:opt_pb}, and motivates
%%   the need for a fast solver.}%
%% \label{fig:maps_tolerance}
%% \end{figure*}

%% In this contribution, we compare a comprehensive list of
%% solvers, all implemented with great care, for solving TV$-\ell_1$
%% regression with a focus on convergence time. First we state the formal
%% problem solved.
%% In section \ref{sec:algos}
%% we present the various algorithms. Experiments done and the results obtained are presented in secions
%%  \ref{sec:experiments} and \ref{sec:results}
%% respectively. Section \ref{sec:discusion} concludes the paper with
%% general recommendations.


%% \section{Formal problem statement and notations}

%% \label{sec:problem_statement}

%% We denote by $\mathbf{y} \in \mathbb{R}^{n}$ the targets to
%% be predicted, and $\mathbf{X}\in\mathbb{R}^{n \times p}$ the
%% brain images related to the presentation of different
%% stimuli. $p$ is the number of voxels and $n$ the number of
%% samples (images). Typically, $p \sim 10^3-10^5$ (for a whole volume),
%% while $n \sim 10-10^3$.  Let $\Omega \subset \mathbb{R}^3$ be
%% the 3D image domain, discretized on a finite grid. The coefficients
%% $\mathbf{w}$ define a spatial map in $\mathbb{R}^p$. Its gradient at a
%% voxel $\omega \in \Omega $ reads:
%% \begin{eqnarray}
%%   \mathbf{\nabla} \mathbf{w}(\omega) := [\nabla_{x} \mathbf{w}(\omega),
%%     \nabla_{y} \mathbf{w}(\omega), \nabla_{z} \mathbf{w}(\omega)] \in
%%   \mathbb{R}^3,
%% \end{eqnarray}
%% where $\nabla_{u}$ is the spatial difference operator along
%% the $u$-axis. Thus $\mathbf{\nabla}$
%% is a linear operator $\in \mathbb{R}^{3p \times
%%   p}$ with adjoint $\mathbf{\nabla} ^T = -\dive \in \mathbb{R}^{p
%%   \times 3p}$. $\Delta := \nabla ^T \nabla \in
%% \mathbb{R}^{p \times p}$ is the Laplace operator. 

%% TV$-\ell_1$ regression leads to the following non-smooth convex
%% optimization problem \cite{gramfort-etal:2013a}:
%% \begin{eqnarray}
%%   \mathbf{\hat{w}} := \argmin_{\mathbf{w}}\;\{E(\mathbf{w}) :=
%%   \mathcal{L}(\mathbf{y}, \mathbf{X}, \mathbf{w}) + \alpha
%%   J(\mathbf{w})\},
%%   \label{eq:opt_pb}
%% \end{eqnarray}
%% where $J(\mathbf{w}) := \theta\|\mathbf{w}\|_{\ell_{1}} + (1 -
%% \theta)\|\mathbf{w}\|_{TV}$ is the regularization and $\alpha \ge 0$
%% controls the amount of regularization. The
%% parameter $\theta \in [0, 1]$, also known as the $\ell_{1}$ \textit{ratio}, is the trade-off between the
%% sparsity-inducing penalty $\ell_{1}$ (Lasso) and TV (isotropic Total
%% Variation):
%% % also known as the $\ell_{1}$\emph{-ratio},
%% % XXX = $\ell_{1}$\emph{-ratio} is sklearn jargon
%% \begin{eqnarray}
%% \|\mathbf{w}\|_{\ell_{1}} := \sum_{\omega \in
%%   \Omega}|\mathbf{w}(\omega)|;\;\;\textbf{ }\|\mathbf{w}\|_{TV} :=
%% \sum_{\omega \in \Omega} \|\nabla \mathbf{w}(\omega)\|_{2} \enspace .
%% %% \\
%% %% &=& \sum_{\omega \in
%% %% \Omega}
%% %% \sqrt{\nabla_{x} \mathbf{w}(\omega)^2 + \nabla_{y} \mathbf{w}(\omega)^2 +
%% %%   \nabla_{z}\mathbf{w}(\omega)^2}\\
%% \end{eqnarray}
%% $\mathcal{L}(\mathbf{y}, \mathbf{X}, \mathbf{w})$
%% is the loss function. Here, we focus on the squared loss
%% and the logistic loss, defined \emph{e.g.} in \cite{michel2011}. The
%% squared loss is natural for regression settings, where $\mathbf{y}$ is a
%% continuous variable, but it may also be used for classification
%% \cite{suykens2002}. The logistic loss is harder to optimize, but
%% more natural for classification settings.


%% \section{Algorithms}
%% \label{sec:algos}

%% In this section, we present the algorithms we benchmarked for solving
%% problem \eqref{eq:opt_pb}.

%% \paragraph{ISTA/FISTA}
%% ISTA~\cite{daubechies2004}, and its accelerated variant
%% FISTA~\cite{beck2009a}, are proximal gradient approaches: the go-to
%% methods for non-smooth optimization. In their seminal introduction of TV
%% for fMRI, Michel \emph{et al.} \cite{michel2011} relied on ISTA.
%% The challenge of these methods for TV is that the proximal operator
%% %% \footnote{The proximal operator (or prox for short) can be seen as a generalization of projection unto convex set.}
%%   of TV
%% cannot be computed exactly; we approximate it in an inner FISTA loop
%% \cite{beck2009b,michel2011}.
%% % , \cite{gramfort-etal:2013a}.
%% %%  and must itself be solved by a
%% %% FISTA\footnote{Note that solving TV and TV$-\ell_1$ are formally very
%% %% close \cite{gramfort-etal:2013a}.
%% % }
%% % \cite{beck2009b,michel2011}. 
%% Here, for all FISTA implementations we use
%% the faster monotonous FISTA variant \cite{beck2009b}. We control the
%% optimality of the TV proximal via its dual gap \cite{michel2011} and
%% use a line-search strategy in the monotonous FISTA to decrease the
%% tolerance as the algorithm progresses, ensuring convergence of the
%% TV-$\ell_1$ regression with good accuracy.

%% \paragraph{ISTA/FISTA with backtracking}
%% A key ingredient in FISTA's convergence is the Lipschitz
%% constant $L(\mathcal{L})$, of the derivative of smooth part of the objective function
%% . The tighter the upper bound used for this constant,
%% the faster the resulting FISTA algorithm. In FISTA, the main use of 
%% $L(\mathcal{L})$ is the fact that: for any
%% stepsize $0 < t \le 1/L(\mathcal{L})$ and for any point $\mathbf{z}$,
%% \begin{equation}%
%%   \begin{gathered}
%%     \mathcal{L}(\mathbf{p}_{t}(\mathbf{z})) \le
%%     \mathcal{L}(\mathbf{z}) + \mathbf{r}_{t}^T\nabla
%%     \mathcal{L}(\mathbf{z}) + \frac{1}{2t} \|\mathbf{r}_{t}\|_{2}^2,
%%     \text{  where} \quad\\
%%     \mathbf{p}_{t}(\mathbf{z}) := \textrm{prox}_{\alpha t J}(\mathbf{z} - t
%%     \nabla \mathcal{L}(\mathbf{z})) \,\text{  and  }\,
%%     \mathbf{r}_{t} :=
%%     \mathbf{p}_{t}(\mathbf{z}) - \mathbf{z} \quad
%%   \end{gathered}
%%   \label{eq:fista_ineq}
%% \end{equation}
%% In least-squares regression, $L(\mathcal{L})$ is precisely the largest
%% singular value of the design matrix $\mathbf{X}$.
%% For logistic
%% regression however, the tightest known upper bound for
%% $L(\mathcal{L})$ is $\|\mathbf{X}\|\|\mathbf{X}^T\|$ (for example see Appendix A of
%% \cite{yuan2012}), which performs
%% very poorly locally (i.e, stepsizes $\sim 1 / L(\mathcal{L})$ are
%% sub-optimal locally). A way to circumvent this difficulty is
%% \emph{backtracking line search} \cite{beck2009a}, where one tunes the
%% stepsize $t$ to satisfy inequality \eqref{eq:fista_ineq} locally at
%% point $\mathbf{z}$. 


%% \paragraph{ADMM: Alternating Direction Method of Multipliers}
%% ADMM is a Bregman Operator Splitting primal-dual method for
%% solving convex-optimization problems by splitting the
%% objective function in two convex terms which are functions of linearly-related auxiliary variables
%% \cite{boyd2010}.  ADMM is particularly appealing for problems such as TV
%% regression: using the variable split $\mathbf{z}
%% \leftarrow \nabla \mathbf{w}$, the regularization is a simple $\ell_{1}/\ell_2$
%% norm on $\mathbf{z}$ for which the proximal is exact and computationally
%% cheap. However, in our settings, limitations of ADMM are:
%% \begin{itemize}
%% \item the $\mathbf{w}$-update involves the inversion of a large ($p \times p$)
%%   ill-conditioned linear operator (precisely a weighted sum of
%%   $\mathbf{X}^T\mathbf{X}$, the laplacian $\Delta$, and the identity
%%   operator).
%% \item the $\rho$ parameter for penalizing the split
%%   residual $\mathbf{z} - \nabla \mathbf{w}$ is hard to set (this is still an
%%   open problem), and though under mild conditions ADMM
%%   converges for any value of $\rho$, the convergence rate depends
%%   on $\rho$.
%% \end{itemize}

%% \paragraph{Primal-Dual algorithm of Chambolle and Pock \cite{chambolle2010}}
%% this scheme is another method based on operator splitting.
%% Used for fMRI TV regression by \cite{gramfort-etal:2013a},
%% it does not require setting a hyperparameter.  % XXX : this is wrong you have the tau / sigma
%% However it is a first-order single-step method and is thus more impacted by the
%% conditioning of the problem. Note that here we explore this primal-dual
%% method only in the squared loss setting, in which the algorithm can be accelerated by
%% precomputing the SVD of $\mathbf{X}$ \cite{gramfort-etal:2013a}
%% .

%% \paragraph{HANSO \cite{lewis2008}}
%% a modified LBFGS scheme based on gradient 
%% sampling methods \cite{burke2005} and inexact
%% line-search. For non-smooth problems as in our case, the algorithm relies on
%% random initialization, to avoid singularities with high probability. 
%% Here, we used the original authors' implementation.

%% \paragraph{Uniform approximation by smooth convex surrogates}
%% $\ell_{1}$ (resp. TV) is differentiable almost everywhere,
%% with gradient $\bigl(\textbf{w}(\omega)/|\textbf{w}(\omega|))_{\omega
%%   \in \Omega}$ (resp. $-\dive (\nabla \mathbf{w}/\|\nabla
%% \mathbf{w}\|_2)$), except at voxels $\omega$ of the weights map where
%% \textbf{w}$(\omega) = 0$ (resp. $\|\nabla \mathbf{w}(\omega)\|_{2} =
%% 0$), corresponding to black spots (resp. edges).  A convenient
%% approach (see for example \cite{NESTA, nesterov2005a, nesterov2005b,
%%   beck2012}) for dealing with such singularities is to uniformly
%% approximate the offending function with smooth surrogates that
%% preserve its convexity. Given  % I removed "and other important properties"
%% a smoothing parameter $\mu > 0$, we define \emph{smoothed} versions
%% of $\ell_1$ and TV:
%% %
%% \begin{eqnarray}
%%   \left.
%%   \begin{aligned}
%%     \|\mathbf{w}\|_{\ell_{1},\mu} &:= \sum_{\omega \in \Omega}
%%     \sqrt{\mathbf{w}(\omega)^2 + \mu^2} \quad\\
%%     \|\mathbf{w}\|_{TV, \mu} &:=
%%     \sum_{\omega \in \Omega} \sqrt{\|\nabla \mathbf{w}(\omega)\|_{2}^2 +
%%     \mu^2} \quad
%%     \end{aligned}
%%     \right\}
%% \end{eqnarray}
%% These surrogate upper-bounds are convex and everywhere-differentiable
%% with gradients that are Lipschitz-continuous with constants $1/\mu$ and
%% $\|\nabla\|^2(1/\mu) = 12 / \mu$ respectively.
%% %
%% They lead to smoothed versions of problem \eqref{eq:opt_pb}:
%% \begin{align}
%%   \mathbf{\hat{w}}_{\mu} := \argmin_{\mathbf{w}}\;
%%   \{E_{\mu}(\mathbf{w}) := \mathcal{L}(\mathbf{y}, \mathbf{X}, \mathbf{w}) + \alpha
%%     J_{\mu}(\mathbf{w})\},
%%   \label{eq:sopt_pb}
%% \\
%%     \text{where} \quad
%%     J_{\mu}(\mathbf{w}) :=\theta \|\mathbf{w}\|_{\ell_{1},\mu} +
%%     (1 - \theta)\|\mathbf{w}\|_{TV, \mu} \quad
%% \end{align}

%% To solve \eqref{eq:opt_pb}, we consider problems of the form
%% \eqref{eq:sopt_pb} with $\mu \rightarrow 0^+$: we start with a coarse
%% $\mu$ ($= 10^{-2}$, \emph{e.g}) and cheaply solve the $\mu$-smoothed problem
%% \eqref{eq:sopt_pb} to a precision $\sim \mu$ using a fast
%% iterative oracle like the LBFGS \cite{ciyou1994}; we
%% obtain a better estimate for the solution; then we decrease $\mu$ by a fixed factor,
%% and restart the solver on problem \eqref{eq:sopt_pb} with this solution; and so on, in a 
%% \emph{continuation} process~\cite{NESTA} detailed in Alg.
%% \ref{Tab:pseudocode_lbfgs}.

%% \begin{algorithm}[htb]
%% \caption{LBFGS algorithm with continuation} \textbf{Let}
%% $\epsilon > 0$ be the desired precision, $\beta$ ($0 < \beta <
%% 1$) be the rate of decay of the smoothing parameter $\mu$, and $\gamma > 0$ be a constant.  Finally,
%% let LBFGS: $(E_\mu, \mathbf{w}^{(0)}, \epsilon) \mapsto \mathbf{w}$ be
%% an oracle which when warm-started with an initial guess
%% $\mathbf{w}^{(0)}$, returns an $\epsilon$-optimal
%% solution (i.e $E_\mu(\mathbf{w}) - E_\mu^{*} < \epsilon$) for problem \eqref{eq:sopt_pb}.\\
%% \textbf{Initialize} $ 0 < \mu^{(0)}$ ($= 10^{-2}$, \emph{e.g}),
%% $\mathbf{w}^{(0)}\in \mathbb{R}^p$, and $k = 0$.\\
%% \Repeat{$\gamma\mu^{(k)}
%%   < \epsilon$}{
%%   \begin{eqnarray*}
%%     \mathbf{w}^{(k + 1)} &\leftarrow& \mbox{LBFGS}(E_{\mu^{(k)}},
%%     \mathbf{w}^{(k)}, \gamma\mu^{(k)})\\
%%     \mu^{(k + 1)} &\leftarrow& \beta \mu^{(k)} \\
%%     k &\leftarrow& k + 1
%%   \end{eqnarray*}
%% } \Return $\mathbf{w}^{(k)}$
%% \label{Tab:pseudocode_lbfgs}
%% \end{algorithm}

%% This algorithm is not faster than
%% $\mathcal{O}(1/\epsilon)$: indeed a good optimization algorithm
%% for the sub-problem \eqref{eq:sopt_pb} is $\mathcal{O}(\sqrt{L_{\mu}/\epsilon})$
%% \cite{nesterov1983}, and $L_{\mu} \sim 1 / \mu \sim 1 / \epsilon$. We
%% believe that this bound is tight but a detailed analysis is
%% beyond the scope of this paper.
  
%% \section{Experiments on fMRI datasets}
%% \label{sec:experiments}
%% We now detail experiments done on publicly available
%% data. All experiments were run full-brain without spatial smoothing.

%% \paragraph{Visual recognition}
%% \label{subsec:haxby}
%% Our first benchmark dataset is a popular block-design fMRI dataset from a study on face and
%% object representation in human ventral temporal cortex  \cite{haxby2001}.
%% It consists of
%% 6 subjects with 12 runs per subject. In each run, the subjects
%% passively viewed images of eight object categories, grouped
%% in 24-second blocks separated by intermittent rest periods. This
%% experiment is a classification task: predicting the object category. We use a
%% two-class prediction target: $\mathbf{y}$ encodes faces versus houses.
%% The design matrix $\mathbf{X}$ is made of
%% time-series from the full-brain mask of $p = 23\,707$ voxels over $n =
%% 216$ TRs, of a single subject (subj1).

%% \paragraph{Mixed Gambles}
%% Our second benchmark dataset is a study in which
%% subjects were presented with mixed (gain/loss) gambles, and decided
%% whether they would accept each gamble \cite{mixedgambles2007}.  No outcomes of these gambles
%% were presented during scanning, but after the scan three gambles were
%% selected at random and played for real money. The prediction task here is
%% to predict the magnitude of the gain and thus a regression on a
%% continuous variable \cite{jimura2012}. The data is pooled across
%% subjects, resulting in 768 samples, each an image of 33\,177 voxels.

%% % We validate our algorithms on both simulated and real data.
%% \smallskip

%% We study the convergence of the algorithms for parameters close to the
%% optimal parameters set by 10-fold cross-validation to maximize prediction
%% accuracy.

%% \begin{figure*}
%% % generate with: ipython ../wip/tv_l1_solver/plot_parallel_plots.py bench/haxby_lr_11th.json .9 --pdb
%% \hspace*{-.01\linewidth}%
%% \includegraphics[width=.6\linewidth]{bench/haxby_lr_energy.pdf}
%% \hspace*{-.09\linewidth}%
%% \includegraphics[width=.6\linewidth]{bench/haxby_lr.pdf}%
%% % \vspace{-2ex}
%% \caption{TV$-\ell_1$ penalized Logistic Regression on the visual
%% recognition face-house
%%   discrimination task. \textbf{Left}: excess energy $E(\mathbf{w}_t) -
%% E(\mathbf{w}_{t})_{t \rightarrow \infty}$ as a function 
%% of time.
%% \textbf{Right}: convergence time of the various solvers for different
%% choice of regularization parameters.
%% Broken lines correspond to a tolerance of $10^{0}$,
%%   whilst full-lines correspond to $10^{-2}$.  The thick vertical line
%% indicates the best model selected by cross-validation.}
%% \label{Fig:HaxbyLR}
%% \end{figure*}

%% %% \paragraph{Haxby 2001: TV$-\ell_{1}$ penalized Least-Squares Regression}
%% %% See figure \ref{Fig:HaxbyMSE}
%% \begin{figure*}
%% % generate with: ipython ../wip/tv_l1_solver/plot_parallel_plots.py bench/haxby_mse_10th.json 1e-2 --pdb
%% \includegraphics[width=.6\linewidth]{bench/haxby_mse.pdf}%
%% \hspace{-.09\linewidth}%
%% % generate with: ipython ../wip/tv_l1_solver/plot_parallel_plots.py bench/poldrack_mse_12th.json 2e1 --pdb
%% \includegraphics[width=.6\linewidth]{bench/poldrack_mse.pdf}
%% % \vspace{-2ex}
%% \caption{TV$-\ell_1$ penalized Least-Squares Regression. \textbf{Left}:
%%   on the visual recognition  face-house discrimination task; \textbf{Right}: on the
%%   Mixed gambles dataset. Broken lines correspond to a tolerance of $10^{0}$,
%%   whilst full-lines correspond to $10^{-2}$. The thick vertical line
%% indicates the best model selected by cross-validation.}
%% \label{Fig:MSEtimes}
%% \end{figure*}

%% \section{Results: convergence times}
%% \label{sec:results}
%% Here, we present benchmark results for our experiments. Figure
%% \ref{Fig:HaxbyLR} gives results for the logistic regression run on the
%% visual recognition dataset: convergence plots of energy as a function of
%% time show that all methods are asymptotically decreasing. Figure
%% \ref{Fig:HaxbyLR} left, shows the time required to give a convergence
%% threshold, defined as a given excess energy compared to the lowest energy
%% achieved by all methods, for different choices of regularization
%% parameters. Similarly, figure \ref{Fig:MSEtimes} shows convergence times
%% for squared loss on both datasets. For these figures,
%% each solver was run for a maximum of 1 hour per problem. Solvers that do
%% not appear on a plot did not converge for the corresponding
%% tolerance and time budget.

%% For logistic loss, the most serious contender is
%% algorithm \ref{Tab:pseudocode_lbfgs}, LBFGS applied on a smooth
%% surrogate, followed by ADMM, however ADMM performance
%% varies markedly depending on the choice of $\rho$. For the squared loss
%% FISTA and algorithm \ref{Tab:pseudocode_lbfgs} are the best performers,
%% with FISTA achieving a clear lead for the larger mixed-gambles dataset.
%% Note that in the case of strong regularization the problem is better
%% conditioned, and first-order methods such as the
%% primal-dual approach can perform well.

%% %% \paragraph{Haxby: TV$-\ell_{1}$ penalized Logistic Regression}
%% %% TV$-\ell_{1}$ penalized Logistic Regression...See figure \ref{Fig:HaxbyLR}


%% \section{Conclusions: approaches to prefer}
%% \label{sec:discusion}

%% TV$-\ell_{1}$ penalized regression for brain imaging leads to
%% very high-dimensional, non-smooth and very ill-conditioned optimization
%% problems. We have presented a comprehensive comparison of state-of-the-art
%% solvers in these setting. Solvers were implemented with all known
%% algorithmic improvements and implementation were carefully profiled and
%% optimized.

%% Our results outline best strategies: monotonous FISTA with a adaptive
%% control of the tolerance of the TV proximal operator, in the case of
%% squared loss; smoothed quasi-newton based on surrogate upper-bounds of
%% the non-smooth norms for logistic loss. While these algorithms are
%% variants of existing approaches, we present here novel additions useful
%% for the TV or TV$-\ell_1$ settings. The fact that the smooth approaches emerge
%% as fast solvers on these non-smooth problems is not unexpected as
%% \emph{i)} the amount of regularization is small \emph{ii)} the prevailing
%% term is smooth and very ill conditioned, thus calling for second-order
%% methods such as Newton or quasi-Newton.

%% For neuroimaging applications, our study has highlighted the need to
%% converge to a good tolerance and the corresponding difficulties. Lack of
%% good solver and explicit control of tolerance can lead to brain maps and
%% conclusions that reflect properties of the solver more than of the
%% TV$-\ell_1$ solution.

%% %\section{Acknowledgements}
%% %% \label{subsec:acknowledgements}
%% %This work was supported by the  Human Brain Project.

%% \medskip \noindent
%% \textbf{Acknowledgments:}
%% This work was supported by the  Human Brain Project.

%% % ==========
%% % = biblio =
%% % ==========
%% % {\small
%% \bibliographystyle{IEEEtran} \bibliography{IEEEabrv,bib_tv}
\end{document}




%
%


